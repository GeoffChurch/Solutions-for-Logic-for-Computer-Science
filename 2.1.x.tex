\subsection*{2.1.1}

Show the following properties:
\begin{enumerate}

    \renewcommand{\labelenumi}{(\alph{enumi})}

    \item If there exists a function $g : B \to A$ such that $f \circ g = I_A$, then $f : A \to B$ is injective.

          \solution[
              By contradiction, assume $f$ is not injective. Then it must collapse two distinct elements into one, i.e. $\exists a_1, a_2 \in A$ such that $f(a_1) = f(a_2) = b$ and $a_1 \neq a_2$. But if $f \circ g = I_A$, then $g(f(a_1)) = g(b) = a_1$ {\bf and} $g(f(a_2)) = g(b) = a_2$. Since $a_1$ and $a_2$ are distinct, $g$ must be non-deterministic, so $g$ cannot be a function as claimed.
          ]

          If $f : A \to B$ is injective and $A \neq \emptyset$, then there exists a function $g : B \to A$ such that $f \circ g = I_A$.

          \solution[
              Let $f^T = \{<b,a> | <a,b> \in f\}$. Since $f$ is injective, $f^T$ is a partial function (deterministic), and since $f$ is total and $f^T$ is a partial function, $f \circ f^T = I_A$.
              \\\\
              Now let $g$ be any total function which extends $f^T$ (this can be done because $f^T$ is a partial function). Since $g$ extends $f^T$, $I_A \subset f \circ g$, and since $g$ is functional/deterministic (in particular on $range(f) = dom(f^T)$) and $I_A$ is total, $f \circ g = I_A$.
          ]

    \item A function $f$ is surjective if and only if there exists a function $g : B \to A$ such that $g \circ f = I_B$.

          \solution[
              Forward direction: for each $b \in B$, let $g(b) = a$ for {\bf some} $a \in f^{-1}(b)$. Since $f$ is surjective, $f^{-1}(b)$ is non-empty, so we can always pick some $a$, and so $g$ is a (total) function. And of course $f(g(b)) = b = I_B(b)$ for all $b \in B$.
              \\\\
              Backward direction: by contradiction, assume $f$ is not surjective. Then there exists $b \in B$ such that $f^{-1}(b) = \emptyset$. Nothing maps to $b$, so $f(g(b)) \neq b$, so $g \circ f \neq I_B$.
          ]

    \item A function $f : A \to B$ is bijective if and only if there is a function $f^{-1}$ called its {\it inverse} such that $f \circ f^{-1} = I_A$ and $f^{-1} \circ f = I_B$.

          \solution[
          Forward direction: given $f$'s injectivity, we know from the problem on injectivity that any functional extension $g$ of $f^T$ satisfies $f \circ g = I_A$, and since $f$ is also surjective the only functional extension is trivially $f^T=f^{-1}$ itself. Furthermore, since $f$ is total and surjective, $f^{-1}$ is as well, and because $(f^T)^T = f$, by symmetry we have $f^{-1} \circ f = I_B$.
          \\\\
          Backward direction: we know from the problems on injectivity and surjectivity that both $f$ and $f^{-1}$ are both injective and surjective, and are therefore bijective.
          ]

\end{enumerate}

\subsection*{2.1.2}

Prove that a function $f : A \to B$ is injective if and only if, for all functions $g,h : C \to A$, $g \circ f = h \circ f$ implies that $g = h$.

\solution[
    Forward direction: by contradiction, let $f$ be injective and $g \circ f = h \circ f$, but assume $g \neq h$. If $g \neq h$ then there exists $c \in C$ where $g(c) \neq h(c)$. But since $f$ is injective then $f(g(c)) \neq f(h(c))$, so $g \circ f$ cannot equal $h \circ f$.
    \\\\
    Backward direction: by contradiction, assume $f$ is not injective, i.e. $\exists a_1, a_2 \in A$ such that $f(a_1) = f(a_2)$ and $a_1 \neq a_2$. Let $g$ and $h$ be indistinguishable except at $c$, where $g(c) = a_1$ and $h(c) = a_2$. Then the difference is washed out by $f$, and we have $g \circ f = h \circ f$ but $g \neq h$.
]

A function $f : A \to B$ is surjective if and only if, for all functions $g,h : B \to C$, $f \circ g = f \circ h$ implies that $g = h$.

\solution[
    Forward direction: by contradiction, let $f$ be surjective and $f \circ g = f \circ h$, but assume $g \neq h$. Then $g(b) \neq h(b)$ for some $b \in B$. Since $f$ is surjective, $f^{-1}(\{b\})$ is nonempty. Pick any element $a \in f^{-1}(\{b\})$. But then $(f \circ g)(a) \neq (f \circ h)(a)$, so $f \circ g \neq f \circ h$.
    \\\\
    Backward direction: by contradiction, assume $f$ is not surjective, i.e. $f^{-1}(\{b\}) = \emptyset$ for some $b \in B$. Let $g$ and $h$ be indistinguishable except at $b$, where $g(b) \neq h(b)$. But then because nothing maps to $b$ and $g$ and $h$ are otherwise identical, $f \circ g = f \circ h$.
]

\subsection*{2.1.3}

Given a relation $R$ on a set $A$, prove that $R$ is transitive if and only if $R \circ R$ is a subset of $R$.

\solution[
    Symbolically, we have (with implicit universal quantification)
    \begin{align}
            R \text{ is transitive iff } & \highlight{xRy \land yRz} \implies xRz \tag{definition of transitivity} \\
            & \highlight{xRy \land yRz} \Leftrightarrow x(R \circ R)z \tag{definition of composition}
    \end{align}

    We can plug the definition of composition into the definition of transitivity (modus ponens) to eliminate $\highlight{xRy \land yRz}$ and derive $$\text{$R$ is transitive iff } x(R \circ R)z \implies xRz$$
]

\subsection*{2.1.4}

Given two equivalence relations $R$ and $S$ on a set $A$, prove that if $R \circ S = S \circ R$, then $R \circ S$ is the least equivalence relation containing $R$ and $S$.

\solution